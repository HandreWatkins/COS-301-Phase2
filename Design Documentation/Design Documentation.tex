\documentclass[12pt]{article}

\usepackage[english]{babel}
\usepackage[utf8x]{inputenc}
\usepackage{pdfpages}
\usepackage{lastpage} % Required to determine the last page for the footer
\usepackage{extramarks} % Required for headers and footers
\usepackage{graphicx} % Required to insert images
\usepackage{listings} % Required for insertion of code
\usepackage{courier} % Required for the courier font

% Margins
\topmargin=-0.45in
\evensidemargin=0in
\oddsidemargin=0in
\textwidth=6.5in
\textheight=9.0in
\headsep=0.25in

\linespread{1.1} % Line spacing

\newcommand{\Title}{Design Documentation} % Assignment title
\newcommand{\Class}{Cos\ 301} % Course/class

\begin{document}

	\vspace{4em}
	
	\begin{center}%
	
	  \LARGE \bf \Title \\[4em]
	  \LARGE {\bf Group 1}\\[1em]
	  \LARGE {\bf Group Members:}\\[2em]
	  \large
	  
	     Mbulungo Musetsho				(10176382) \\[1em]
	     Ndivhuwo Ntambeleni			(10001183) \\[1em]
	     Pule Legodi                                (29302732) \\[1em]
	    	%Enter your details below just as the one above
	    
	\end{center}%
	

	\newpage
	\tableofcontents
	
	\newpage
	\section{Background}
	
		\vspace{0.2in}
	
				 %We must give an introduction of this document
		
		 
	
	\section{Vision}
	The purpose of the project is to develop a software solution which provides
a web and mobile platform for markers, students and lecturers handle marks
and marking in the Department of Computer Sciences. It will uphold the
privacy of students so that students cannot see each others marks. It will
reduce paper work and the chances that marks can be lost and provide a
centralized repository for student marks. It will provide functionality to
generate customizable reports at different levels of granularity.
		\vspace{0.2in}
		
			 		 %We must briefly describe the purpose of this document
				
	
	\section{Software Architecture Design}
			%This section needs to demonstrate how the system will be able, within designed software architecture, to address the non-functional requirements.
		\vspace{0.2in}
		
		\subsection{Choices of Technologies}
		%Please refer to the guideline document for some of the items that must be included in this part.
			\vspace{0.2in}
		
		
		\subsection{Chosen Frameworks}
				%Please refer to the guideline document for some of the items that must be included in this part.
					\vspace{0.2in}	
			
		
		\subsection{Chosen Protocols}
				%Please refer to the guideline document for some of the items that must be included in this part.
			\begin{enumerate}
	                    \item Simple Object Access Protocol (SOAP) \\*
	                    All systems (devices) should be able to access the system’s content through the SOAP-based web services. 
	                        \begin{enumerate}
	                            \item The devices (systems) will send a request, procedure call, or a message such as getMark to search or retrieve information (average mark for the module, number of distinctions) through the SOAP-based web services.
	                            \item The SOAP-based web service will process the request or message and send a response to the devices (systems) in a form of a structured XML-based data.
	                        \end{enumerate}
	                    \item Lightweight Directory Access Protocol (LDAP)
	                        \begin{enumerate}
	                            \item It is used to authenticate the system users (Lecturers, Students, and Teaching Assistants) on login for single Sign On.
	                            \item It is used to retrieve the student’s personal details and class lists, including courses assigned to lectures and students.
	                            \item It is used to retrieve system users roles such as, is the user a Lecturer, Student, or teaching assistant for the Module
	                        \end{enumerate}
	                    \item  Hypertext Transfer Protocol over Secure Socket Layer (HTTPS and HTTP)
	                        \begin{enumerate}
	                            \item Used to send and retrieve sensitive data(marks and modules) through a secured (HTTPS) channel.
	                            \item Used to send a GET and POST requests to the web server and web services. Such as adding or updating a student mark.
	                        \end{enumerate}
	                        
              		\end{enumerate}
			\vspace{0.2in}
			
		
		\subsection{Chosen Libraries}
				%Please refer to the guideline document for some of the items that must be included in this part.
			\vspace{0.2in}

			
	\section{Application Design}
				%must address functional requirements as specied in re-quirements specification.
			\vspace{0.2in}
					
		\subsection{Lower Levels of Granularity Specification}
				\begin{itemize}
					\item User Log In
						\begin{itemize}
						\item View: A log in screen is displayed, wherein the user enters their details then submit to log in. The user will receive a log in result based on the validity of their credentials
						\item Controller: The Controller receives the credentials entered by the user and transfers them to the model for validation. The controller will create a response object based on the validity of the credentials.
						\item Model: The model receives the credentials and simply finds a match from LDAP. It sends the result of the validation back to the controller.
						\end{itemize}
						
					\item Leaf Assessments
					\begin{itemize}
						\item View: On the displayed screen, the user selects one of the following:
						\begin{enumerate}
							\item Add New Leaf Assessment: The user enters all the required information to create a leaf assessment then submits. This will cause a new Assessment to be created and saved into the database, unless there was an error or exception (in which case an exception is thrown and the user is informed of the error)
							\item Search Existing Leaf Assessment (for Update/Delete): If record(s) matching the search were found, they are displayed for the user to update or delete, else then the user is informed of anything that possibly went wrong.
						\end{enumerate}
						\item Controller: The controller receives, based on the kind of operation the user selects, the required information and sends it to the Model for either insertion, update, or deletion.
						\item Model: The model will respectively receive the required information and either save a new leaf assessment, update or delete an existing leaf assessment. After the respective opration, the model will notify the view of changes made to the database.
						 
					\end{itemize}
					\item Aggregate Assessment:
						\begin{itemize}
							\item View:The user enters all Aggregate Assessment information as required. When they submit, a new Aggregate Assessment is create based on the specified information.
							\item Controller: The controller uses the provided aggregator to proccess the actual aggregate assessment then it sends it to the Model for storage.
							\item Model: The model receives the aggregate assessment information and provides the controller with the specified aggregator. Then the model receives the processed aggregate assessment and stores it into the database then notifies the user about the results of the operation.
						\end{itemize}
					\item Assessment Reports
						\begin{itemize}
							\item View:The user specifies all required information (including the Freequency Analysis Information) and then submit. They will then choose whether they want it rendered on their device application or downloaded in either PDF or CSV format. Based on their choice of view, the results will be displayed.
							\item Controller: The controller retrieves the report specification submitted by the user and sends it to the model to find matches. The controller will then create a response object with the results and send it to the user for viewing.
							\item Model: The model receives the report specification from the controller and then searches against the database for possible matches then it returns the matches back to the controller.
						\end{itemize}
					\item Student Marks Reports
						\begin{itemize}
							\item View:The user specifies all required information (including aggregation specifications) and then submit. They will then choose whether they want the results rendered on their device application or downloaded in either PDF or CSV format. Based on their choice of view, the results will be displayed.
							\item Controller: The controller retrieves the marks report specification submitted by the user and sends it to the model to find matches. The controller will then create a response object with the results and send it to the user for viewing.
							\item Model: The model receives the marks report specification from the controller and then searches against the database for possible matches then it returns the matches back to the controller.
						\end{itemize}
					\item Audit Reports
						\begin{itemize}
							\item View: The user enters all the specifications for the Audit Report then submit. They will then choose whether they want the results rendered on their device application or downloaded in either PDF or CSV format. Based on their choice of view, the results will be displayed.
							\item Controller: The controller retrieves the Audit Report specifications submitted by the user then send it to the Model to retrieve matches. These matches will be sent back to the controller and then sent back to the user for viewing.
							\item Model: The model receives the Audit Report specifications, retrieves matches of these specifications, and then return them to the controller.
						\end{itemize}
				\end{itemize}
		\vspace{0.2in}
		
		\subsection{API Specifications}
						%UML interfaces and class diagrams for inputs and outputs, etc
						\begin{itemize}
				\item Different API's are to be used throughout the whole project, these are summarised below according to the interface they fall under: 
					\begin{itemize}
                                                \item View: Android API, HTTP/HTML 5 API, CSS API.
						\item Controller: Python Django API (Main API). (PHP API and JAVA API might be used under certain conditions).
						\item Model: mySQL API, LDAP API.
					\end{itemize}
			
		\end{itemize}
				\vspace{0.2in}
				
		\subsection{System Class Diagrams }%Ndivhuwo Nthambeleni
		%Class Diagram for assessment marking and assessment aggregation
				
				\begin{itemize}
                                        \item Leaf Assessment, Mark allocation and assessment aggregation Class diagram
                                                \begin{itemize}
                                                \item The class diagram uses the abstact factory design pattern to represent the lower level relationship between an assessment, it's aggregation and the mark allocation.
                                                \item Although all these processes are done in different APIs, this class diagram represents them in a manner that seems like they are on one interface just to make it better to understand.
                                                \end{itemize}
                                        \begin{figure}[h]
                                                \centering
                                                \includegraphics[width=6in, height=4in]{Pictures/MiniPhase2ClassDiag.jpg}
                                                \caption{Assessment Marking and aggregation Class Diagram}
					\end{figure}        
                                        \item Assessment Report Class Diagram
						\begin{itemize}
                                                \item Just like the previous class diagram, this the assessment report class diagram uses the abstract factory design pattern to show how each report is related to a specific assessment.      
                                                
						\end{itemize}
					\begin{figure}[h]
                                                \centering
                                                \includegraphics[width=6in, height=4in]{Pictures/assessmentReportClassDiag.jpg}
                                                \caption{Assessment Report Class Diagram}
					\end{figure}
                                        \item Student Mark Report Class Diagram:
                                                \begin{itemize}
                                                        \item This is an abstract factory class diagram, like the others
							\item Shows how the marks are generated for each student when they are logged on to the system.
						\end{itemize}
					\begin{figure}[h]
                                                \centering
                                                \includegraphics[width=6in, height=4in]{Pictures/StudentMarksReport.jpg}
                                                \caption{Student Mark Report Class Diagram}
					\end{figure}
                                        
                                \end{itemize}
						%These class diagrams must include attributes, methods and relationships
						
				\vspace{0.2in}
		
		\subsection{System Process Specification}%Mbulungo Muusetsho
						%Sequence and activity diagrams for detailed system process specifications
						\begin{figure}[h]
										\centering
										\includegraphics[width=6in, height=4in]{Pictures/LoginActivityDiagram.jpg}
										\caption{User Log In Activity Diagram}
						\end{figure}
						\begin{figure}[h]
										\centering
										\includegraphics[width=6in, height=4in]{Pictures/LeafAssesmentActivityDiagram.jpg}
										\caption{Leaf Assessment Activity Diagram}
						\end{figure}
						\begin{figure}[h]
										\centering
										\includegraphics[width=6in, height=4in]{Pictures/AggregateAssesmentActivityDiagram.jpg}
										\caption{Aggregate Assessment Activity Diagram}
						\end{figure}
						\begin{figure}[h]
										\centering
										\includegraphics[width=6in, height=4in]{Pictures/AssessmentReportActivityDiagram.jpg}
										\caption{Assessment Report Activity Diagram}
						\end{figure}
						\begin{figure}[h]
										\centering
										\includegraphics[width=6in, height=4in]{Pictures/StudentMarksReport.jpg}
										\caption{Student Marks Report Activity Diagram}
						\end{figure}
						\begin{figure}[h]
										\centering
										\includegraphics[width=6in, height=4in]{Pictures/AuditReportActivityDiagram.jpg}
										\caption{Audit Report Activity Diagram}
						\end{figure}
				\vspace{0.2in}
		
		\subsection{User Interface Designs} %MatthewHughes and Ndivhuwo Nthambeleni
						%This includes work-flow specifications
				\vspace{0.2in}
		
		\subsection{Database Desgin} %Lutfiya
						%Database tables, etc
						
				\vspace{0.2in}
			
			
	
	\section{Open Issues}
	
		\vspace{0.2in}
		
		
	\newpage
	\section{Glossary}
	
		\vspace{0.2in}
		
			
	
	
\end{document}
